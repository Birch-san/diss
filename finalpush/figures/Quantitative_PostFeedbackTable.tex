\renewcommand*{\arraystretch}{1.5}
\begin{longtable}{p{1cm}|p{6in}}
\hline\hline
Which ID were you allocated?	&	Have you any other feedback?	\\ [0.5ex]
\hline
\endhead % all the lines above this will be repeated on every page
1	&	Sometimes when using flyouts on an area with many links, the suggestions would become a bit cluttered making it slightly difficult to see what I was pointing at. However, by choosing flyouts, this is a sacrifice I am willing to make (and understand why).\newline
Also, when performing tabbing on the amazon page, I had very little knowledge about the current `location' of the tab. Even on websites that did show the location, it was often just a very faint dotted border, which really isn't much easier to see.\newline
Additionally, tabbing involved holding down of keys, which is less physical exertion than the counted keypresses might imply.	\\
2	&	Flyouts required too much additional time to parse what to press, had to think more about pressing fewer keys.	\\
3	&	great work	\\
4	&	Maybe an extension to these (mainly flyouts) would be to have a priority system in the background that chooses which keys to highlight based on previous browsing history rather than what seems like random. This could be applied to crosshair as well as tabbing in some form.\newline
Links weren't entirely that clear using flyout as colours and positions didn't necessarily match causing some confusing.\newline
Additionally, if keyboards don't have keypads (like some small laptops) this can make the usage of crosshair and flyouts harder as the screen division do not map directly to the keyboard layout.	\\
5	&	It was fun! The chibipoint system seems very useful, for those with/without disability. It enables faster pin-pointing of data, making use of the time spent on the Internet more efficient.	\\
7	&	Found ChibiPoint with Flyouts easier perhaps because by this  point in the study I was used to the Crosshairs method of navigation. My opinion may be different if I had done the tests in a different order.\newline
I generally found the Flyouts version quite useful, however question the effectiveness of the use of colours to highlight the link suggestions. On a webpage with a lot of colours this just because confusing, especially when the key shortcuts to the Flyout highlighted links overlapped and the shortcut appeared over the face of other links. Thus the shortcuts were a little misleading as one could assume that the link that the shortcut is positioned over is the link that shortcut would activate.\newline
It was sometimes a little confusing to work out which quadrant was the best to use for Crosshairs when the link falls very close to the bottom line in the row and overlaps between two adjacent quadrants. 	\\
6	&	Flyouts need to be more spread out. Colour coordination is good but when there are quite a few links close together, it was a bit hard to see which key to press as they were clumped together. Overall I thought the system was very intuitive and was simple to use. Also, it mapped well to the Numpad which made navigation easier and more `familiar' so to speak. 	\\
8	&	Too many key presses for tabbing and in some cases, e.g. Amazon, the button's were not highlighted so it required some careful investigation to find out what I was about to press.\newline
Chibi with just crosshairs was unique and I found it efficient, though having the html container light up was confusing.\newline
Chibi with flyovers [sic] worked great!  The only problem was that sometimes, especially when the buttons where close together, finding the colour of the highlighted button and its highlighted key was confusing as they were too cluttered.	\\
9	&	Numpad is very handy for the grid interface and touch typing. It would be tricky to use the number row on a laptop keyboard, though.	\\
10	&		\\
11	&	Tabbing is occasionally suitable in some special cases, where things happen to work out just right. However ChibiPoint with flyouts seems to work just as well and often better in all cases. Without flyouts misses can occasionally happen unexpectedly, but with flyouts, the offered `guesses' at what the user wants to click are very good. These guesses make it very clear what will be clicked on and can further reduce the number of clicks needed to navigate a page. 	\\
12	&	Chibi point was far less variable than tabbing. Tabbing was easier for large websites with not many controls but for websites with lots of different controls chibi point was much better. Flyouts were helpful most of the time but occasionally obscured the control I was trying to press.	\\
\hline
\caption{(Continued) Post-experiment questionnaire responses of Quantitative Study participants}
\label{fig:quantpost_feedback}
\end{longtable}